%%%==============================================================================
%%
%% Auxiliary (sample) local.tex file
%%
%%%==============================================================================
%

\ActivitySetCoord[course]{Fabiano Disconzi Wildner}[m]
% o nome do curso pode ser redefinido (ex. para Monografias)
%\ActivitySet{CCA}{Engenharia de Controle e Automação}


\ActivitySetCoord[internship]{Rafael Antônio Comparsi Laranja}[m]

\ActivitySetCoord[tccI]{Lucíola Campestrini}[f]

\ActivitySetCoord[tccII]{Lucíola Campestrini}[f]

%%
%%Outro elementos podem ser ajustados, se necessário
%%
%\SetHeadings{%
%    university              = {Universidade Federal do Rio Grande do Sul} ,
%    acronym                 = {UFRGS} ,
%    unit                    = {Escola de Engenharia} ,
%    course                  = {Engenharia de Controle e Automação} ,
%    course.title            = {Bacharel em Engenharia de Controle e Automação} ,
%    departament             = {Depto. de Sistemas Elétricos de Automação e Energia} ,
%}

% no caso de um relatório de disciplina
%\reportclass{ENG10004}{Sistemas de Controle I}
%\reportdescription{Este relatório apresenta os resultados obtidos pelos autores no desenvolvimento da segunda tarefa da disciplina.}


% o local de realização do trabalho pode ser especificado (ex. para Monografias)
%\location{São José dos Campos}{SP}

%%%%%%%%%%%%%%%%%%%%%%%%%%%%%%%%%%%%%%
%%%%%%%%%%%%%%%%%%%%%%%%%%%%%%%%%%%%%%
%%%%%%%%%%%%%%%%%%%%%%%%%%%%%%%%%%%%%%

\student{Rodrigues da Silva}{Maxwell}[m]
\studentinfo{282162}{maxwell.silva@ufrgs.br}

% alguns documentos podem ter varios autores:
%\student{Formando}{Nome do Aluno}[m]
%\studentinfo{No. cartão2}{email}


% Informações gerais
%
\worktitle{Aplicação e Avaliação de Machine Learning em Dispositivos Embarcados}

% orientador
\advisor[Prof.~Dr.]{Götz}{Marcelo}[m]
\advisorinfo{UFRGS}{Doutor pela Universidade Federal do Rio Grande do Sul -- Porto Alegre, Brasil}{mgoetz@ufrgs.br}{3338}

\tutor[Prof. Dr.]{do Tutor}{Nome}[m]
\tutorinfo{UFRGS}{Instituição -- Cidade, País}{email}{ramal}

\supervisor[Eng. Mec/Eletr/Ctrl]{do Supervisor}{Nome}[m]
\supervisorinfo{crea}{posição/cargo}{email}{ramal}

\internship{Empresa Ltda.}{P\&D}{10/10/22}{20/12/22}{2 Meses}


% obviamente, o coorientador é opcional
%\coadvisor[Prof. Dr.]{do Coorientador (se houver)}{Nome}[m]
%\coadvisorinfo{UFRGS}{Doutor pela (Instituição onde obteve o título -- Cidade, País)}{email}{ramal}

% banca examinadora
\examiner[Prof.~Dr.]{Laranja}{Rafael Antônio Comparsi}[m]
\examinerinfo{UFRGS}{Doutor pela Universidade Federal do Rio Grande do Sul -- Porto Alegre, Brasil}{rafael.laranja@ufrgs.br}{3749}

\examiner[Prof.~Dr.]{Brusamarello}{Valner João}[m]
\examinerinfo{UFRGS}{Doutor pela Universidade Federal de Santa Catarina -- Florianópolis, Brasil}{valner.brusamarello@ufrgs.br}{4475}


% suplentes da banca examinadora (apenas para alguns formulários)
%\altexaminer[Prof.~Dr.]{do professor suplente I)}{(nome}[m]
%\altexaminerinfo{sigla da Instituição I onde atua}{Doutor pela (Instituição Ia onde obteve o título -- Cidade, País)}{email}{ramal}


%% resumo do trabalho (para o formulário de renovação de requerimento de matrícula.
%%
\workbrief{algo a ser feito...}
\advisorreview{Parecer final do Orientador.}
\coadvisorreason{justificativa para ter-se um co-orientador...}
\workchange{Justificativa para mudança de tema em TCC II}

%%\SetTerms{
%%	tccI.text-descpage = {Seu novo texto para a página de descrição do TCC I.}
%%}

%%\SetTerms{
%%	tccI.text-approvalpage = {meu texto personalizado hehehe.}
%%}

% a data deve ser a da defesa; se nao especificada, são usados
% mes e ano correntes
\pubdate{Junho}{2025}


